\section{Завдання 1}

\begin{question}
    Research has shown that in a certain language, the distribution of the number of letters in words in texts is close to Poisson with parameter 4. 
        
    \begin{enumerate}
        \item LOL KEK
        \item LOL KEK
        \item LOL KEK
    \end{enumerate}
        
    Use the central limit theorem to approximate the probability that a text of 1000 words has more than 4100 letters. Explain explicitly what assumptions you are making to guarantee that
    \begin{itemize}
        \item LOL KEK
        \item LOL KEK
        \item LOL KEK
    \end{itemize}
    this approximation can be used.
\end{question}

\begin{answer}
    We follow the four steps:
    \begin{enumerate}
        \item We need to assume that subsequent words are i.i.d., and that their expectation and variance are bounded.
        
        \item The lengths of words are Poisson with parameter 4, so that the mean and the variance are 4.
        
        \item We are asked to compute the probability that a text of 1000 words has more than 4100 letters: $\mathbb{P}(S_{1000} > 4100)$, where $S_{1000}$ is the number of letters in the text of 1000 words. 
        
        We rewrite this as
        \begin{equation*}
            \mathbb{P}(S_{1000} > 4100) = \mathbb{P}\left( \frac{S_{1000} - 4100}{\sqrt{1000 \cdot 4}} > \frac{4100 - 4000}{\sqrt{1000 \cdot 4}} \right) \approx \mathbb{P}(Z_{1000} > 1.58112).
        \end{equation*}
        
        \item CLT approximation: We approximate
        \begin{equation*}
            \mathbb{P}(S_{1000} > 4100) \approx \mathbb{P}(Z_{1000} > 1.58112) \approx \mathbb{P}(Z > 1.58112) \approx 0.0571.
        \end{equation*}
    \end{enumerate}
\end{answer}